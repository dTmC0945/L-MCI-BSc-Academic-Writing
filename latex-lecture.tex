%%% latex-lecture.tex --- practical latex guide
%
% Copyright (c) 2025 DTMc

% ---------------------------------------------------------------------------------------------

% To start with any type of document we need to have a `documentclass'.  a `documentclass' is a
% command which houses all the necessary information about how the document should be
% formatted. This command tells us a couple of things:
%
% 1. How should the margins be?
% 2. Defining the sectioning commands.
% 3. Describing the useful environments.
%
% Here we add the following optionals (i.e., items in square brackets [ ] separated by by comma
% (,) ).  Let's have a look at the options we have given in the class
%
% a4paper : Here we ask the document page size to be that of A4,
% 10pt    : Here we ask the document to have a font size of 10pt.
%
% Here the unit `pt' is defined as point and the conversion is (those of you who are
% interested) is:
%
% 72 pt = 1 inch = 2.54 cm
%
% Finally here we describe the `type' of the document we are working. The standard LaTeX
% supports three (3) types out-of-the-box:
%
% book    : for long documentation work (i..e, book, thesis)
% article : for writing short works (i.e., academic paper)
% report  : for working with long(ish) documents (technical reports)
\documentclass[a4paper, 10pt]{book}

% PREAMBLE ------------------------------------------------------------------------------------

% Once we have described our document let's add some sugar to it and use additional
% packages. When it comes to anything an important principle to use is `DO NOT REINVENT THE
% WHEEL' and this also applies to working with LaTeX. There are over 1000 packages available
% for you to use in your work, and the best part is, because the work is open-source, you can
% use it in your work.
%
% Let's load some packages
%
% Starting with `siunitx'. This package allows you to write units consistently, meaning the
% spacing within units and how their font will be represented will always be correct.
\usepackage{siunitx}

% \setcounter{secnumdepth}{0}

\usepackage{lipsum} 
\usepackage{xcolor}
\usepackage{enumitem}

% For typesetting mathematical equations.
\usepackage{amsmath}

\definecolor{mcitest}{HTML}{234fa3}
%opening
\title{First Lecture in Latex}
\author{Daniel}


% \setlength{\marginparwidth}{15cm}

% BEGINNING OF THE DOCUMENT -------------------------------------------------------------------

% For us to work with LaTeX, as I have mentioned we need to tell what kind of document we are
% working. Here we may have chosen book, article, or report. Once chosen we still need to tell
% where the document is being written. For that we need to wrap all the content we are writing
% in an `environment', which is of the following form:
%
% \begin{document}
% ...
% \end{document}
%

\begin{document}

% A useful feature of LaTeX is the generation of Table of Contents is handled automatically
% by the use of the command `\tableofcontents'.
\tableofcontents

% These commands might be needed if the content is stretched through the page. The following
% commands basically does the following:
% `\vspace*{\fill}' : Creates and infinitely stretchable invisible space so the content in
%                     the page is not stretched.
% `\pagebreak'      : Tells latex that the page is finished and proceed onto the next page.
\vspace*{\fill}
\pagebreak

% To start working properly, lets begin with a chapter.  The table has each SECTIONING-COMMAND
% in LaTeX.  All are available in all of LaTeX's standard document classes `book', `report',
% and `article', except that `\chapter' is not available in `article'.
% 
% Sectioning unit    Command            Level
% --------------------------------------------------------------------
% Part               `\part'            -1 (`book', `report'), 0
%                                       (`article')
% Chapter            `\chapter'         0
% Section            `\section'         1
% Subsection         `\subsection'      2
% Subsubsection      `\subsubsection'   3
% Paragraph          `\paragraph'       4
% Subparagraph       `\subparagraph'    5
\chapter{First Steps in \LaTeX}

To start let's use a very famous passage from a very famous book:

% Please observe that the each empty space between paragraphs tells latex to break the content
% into a new paragraph. To tell that it is a new paragraph Latex adds and `indent'. Of course
% if we want to disable the indentation, all we have to do is write `\noindent' at the
% beginning of the paragraph

It was the best of times, it was the worst of times, it was the age of wisdom, it was the age
of foolishness, it was the epoch of belief, it was the epoch of incredulity, it was the season
of Light, it was the season of Darkness, it was the spring of hope, it was the winter of
despair, we had everything before us, we had nothing before us, we were all going direct to
Heaven, we were all going direct the other way--in short, the period was so far like the
present period, that some of its noisiest authorities insisted on its being received, for good
or for evil, in the superlative degree of comparison only. \\

% Please observe that at the end of the aforementioned paragraph we have added a `\\'. This
% forces the content to be finished and added an additional empty line.

\noindent It was the best of times, it was the worst of times, it was the age of wisdom, it was
the age of foolishness, it was the epoch of belief, it was the epoch of incredulity, it was the
season of Light, it was the season of Darkness, it was the spring of hope, it was the winter of
despair, we had everything before us, we had nothing before us, we were all going direct to
Heaven, we were all going direct the other way--in short, the period was so far like the
present period, that some of its noisiest authorities insisted on its being received, for good
or for evil, in the superlative degree of comparison only.

% Finally if we want to have NO indentation across the document we just add
% \setlength{\parindent}{0pt}
% to the preamble. The reason we write it in the preamble is that the code affects all
% the indentation below and to make sure it affects the whole document, we place it
% in the preamble.

% Let's now write a section which if we recall is one level below the `chapter'. Please
% observe the numbering of the section. In a chapter the number is a single value whereas
% in a section the numbering is two. The first digit represents the chapter and the second
% digit represents the section value.

\section{A Gentle Introduction}

% If we want to have this section unnumbered (i.e., have no number on its left and have no
% mention of it in the table of contents) we just add an asterisks (*) shown as:
%
% \section*{A Gentle Introduction}

\subsection{A Letter of Introduction}

% Let's write a nice introductory text. We start with a `center' environment which as the name
% suggest is an environment where all the content it contains is centered. Here we also use the
% command `\\' but here, if you notice, we added a spacing optional value of `[0.75cm]' value.
% this tells latex how much spacing will be forced once the line is broken.

\begin{center}
	MCI Mechatronics \\[0.75cm]
	Welcomes You!
\end{center}

% Let's write a nice line with no indent and also have some colour. In latex colour does not
% come out of the box so we need to use a package called
%
% \usepackage{xcolor}
%
% in our preamble. Once done we can use the command with the following syntax:
%
% `\textcolor{<colour>}{<text to be colourised>}'
%
% Here we are using the colour `mcitest' we have defined in the preamble.

\noindent This is \textcolor{mcitest}{to certify} that you have chosen to be engineers by the
times your degree is finished!!

% The `flushright' environment pushes all the text to be right-aligned.
\begin{flushright}
	The Lecturer \\
	D. T. McGuiness
\end{flushright}

\subsection{Font modifiers}

The following is a showcase of how fonts can be changed in \LaTeX.

\noindent This is a \textbf{bold} text. \\ % Create bold text
This is \texttt{Hello, World!} example. \\ % Create typewriter (i.e., code) text
This is \textit{Italic} text. \\           % Create italic text
This is \textsl{slanted} text. \\          % Create slanted text 
This is \textsc{small caps} text. \\       % Create small-caps text 
This is \textsf{Sans serif} text. {\marginpar{\footnotesize\begin{flushleft}
				This is a margin text that I am currently writing to I can look back at it later
			\end{flushleft}}}\\


% The following standard type size commands are supported by LaTeX.  The
% table shows the command name and the corresponding actual font size used
% (in points) with the `10pt', `11pt', and `12pt' document size options,
% respectively (*note Document class options::).
% 
% Command                     `10pt'    `11pt'    `12pt'
% --------------------------------------------------------
% `\tiny'                     5         6         6
% `\scriptsize'               7         8         8
% `\footnotesize'             8         9         10
% `\small'                    9         10        10.95
% `\normalsize' (default)     10        10.95     12
% `\large'                    12        12        14.4
% `\Large'                    14.4      14.4      17.28
% `\LARGE'                    17.28     17.28     20.74
% `\huge'                     20.74     20.74     24.88
% `\Huge'                     24.88     24.88     24.88
%
% We put the curly brackets (i.e., { } ) to limit the effect of the font modifiers. If the
% brackets were not present every content after the invocation of the command would be changed.

{\Huge This is a Huge Text!!} and this is normalsize \textcolor{green}{This is something I need
	to look back!!!}

\paragraph{This is a paragraph} And I can continue on the same line. Paragraph is unique in the
sense that it does not cause the line to be broken but allows the content to continue on the
line itself. This is generally useful if we want to emphasise the content.

% Finally as a gentle anectode we can use the `quote' or `quotation' environments to show
% quotes within the document.

% Both environments indent margins on both sides by `\leftmargin' and the text is
% right-justified.
% 
% They differ in how they treat paragraphs.  In the `quotation' environment, paragraphs are
% indented by 1.5em and the space between paragraphs is small.  In the `quote' environment,
% paragraphs are not indented and there is vertical space between paragraphs.

\begin{quote}
	\lipsum[1-2]
\end{quote}

\section{Working with Lists}

% To produce a list we need to use the `itemize' environment. This produces an ``unordered
% list'', sometimes called a bullet list.  There must be at least one `\item' within the
% environment; having none causes the LaTeX error:

% Something's wrong--perhaps a missing \item

Let's write a list

\begin{itemize}
	\item This is an item,
	\item So is this one,
	\item and why not a third one but this time the line is much longer than it should be
	      %
	      \begin{itemize}
		      \item this is a sub item
		      \item this is another one
	      \end{itemize}
	      %
\end{itemize}

Let's do an enumerated list. To change the behaviour of the \texttt{enumerated} list we use
\texttt{enumitem} package. Change the numbering by using label.

\begin{enumerate}[label=(\Roman*).]
	\item This is my first item,
	\item This is my second item,
	\item and this is my final item.
	      %
	      \begin{enumerate}
		      \item This is a sublist,
		      \item and this is my second item within the sub list
	      \end{enumerate}
	      %
\end{enumerate}

If I want to continue the numbering I use \texttt{resume} option.

\begin{enumerate}[label=(\Roman*)., resume]
	\item Now we continue where we left off.
\end{enumerate}

\subsubsection{Description Lists}

Used for describing items,

\begin{description}
	\item[LaTeX] A very powerful text description language.
\end{description}

\subsubsection{Verbatim Environment}

Allows you to print without worrying about escape characters.

\begin{verbatim}
	This text will be printed as is without any
	modification. I can use control characters (\,$,%)
	without any problems.
\end{verbatim}


\chapter{Typesetting Mathematics}

A straight line is represented with $ax + by + c =0$, where $a$, $b$, and $c$ are
\textbf{constants}.

We can also use \texttt{equation} environment as well.

\begin{equation}
	x^{n + 2} + y^{n} = z^{n} \quad n>2
\end{equation}

To use \textbf{superscripts} use \textasciicircum, and to use \textbf{subscripts} use \_

\begin{equation}
	x_{n} + y_{n} = z_{n}
\end{equation}

We can also use both at the same time

\begin{equation}
	x_{n}^{b} + y_{n}^{a} = z_{p}^{g}
\end{equation}

You can have superscripts of superscripts

\begin{equation}\label{eq:supersuper}
	{{x^{2}}^{3}}^4 \qquad {{y_{2}}_{3}}_{4}
\end{equation}

I am referencing equation \ref{eq:supersuper}.

\begin{equation*}
	\int_{0}^{100}x^{2}\,dx = A
\end{equation*}

\end{document}


% ---------------------------------------------------------------------------------------------
%
% latex-lecture.tex ends here
